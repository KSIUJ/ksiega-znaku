\documentclass[titlepage]{mwart}
\usepackage[utf8]{inputenc}
\usepackage{polski}
\usepackage[explicit]{titlesec}
\usepackage{chngcntr}
\usepackage{graphicx}

\renewcommand{\thesection}{\Roman{subsection}}
\titleformat{\section}{\centering\normalfont\Large\bfseries}{}{0em}{\thesection #1}
\newcommand{\lexsection}[1]{\section{#1}}

\titleformat{\paragraph}{\centering\normalfont\large\bfseries}{}{0em}{#1 \theparagraph}

\title{Księga Znaku Koła Studentów Informatyki\\ Uniwersytetu Jagiellońskiego}
\date{2018\\ Grudzień}
\author{Jakub Kiermasz}

\begin{document}
\maketitle

\lexsection{Geneza}
Zadaniem niniejszej Księgi jest zdefiniowanie wszystkich logo oraz pieczęci Koła Studentów Informatyki Uniwersytety Jagiellońskiego (zwanego dalej Kołem). Ma to na celu stworzenie oraz utrzymanie dobrego i jednolitego wizrenku Koła.
\\
\newline
Jednyą z głównych wytycznych podczas tworzenia tej księgi było odświeżenie i dopracowanie znaków graifcznych przy jednoczesnym zachowaniu ważnych historycznie parametrów.

\lexsection{Podstawowa forma znaku}
Logo podstawowe składa się z okręgu oraz litery ksi, dwóch gwiazdek wraz z częścią literniczą, która stanowi napis "Koło Studentów Informatyki Uniwersytetu Jagiellońskiego", wpisanej wewnątrz okręgu. Litera ksi nawiązauje do akronimu KSI. 

\vspace{2\baselineskip}
\begin{center}
   \includegraphics[width=0.5\textwidth]{logo-ksi-basic.png} 
\end{center}


\end{document}

